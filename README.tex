% Created 2018-04-03 Tue 11:13
% Intended LaTeX compiler: pdflatex
\documentclass[11pt]{beamer}

      \usepackage{pgfpages}

      %\setbeameroption{show notes}

      %\setbeameroption{show notes on second screen=right}

      \mode<presentation>

      \usetheme{Madrid}

      %\usecolortheme{{{{beamercolortheme}}}}

      \setbeamercolor{structure}{fg=cyan!50!black}
      \setbeamercolor*{block title example}{fg=cyan!50!black,bg=gray!10}
      \setbeamertemplate{navigation symbols}{} % to remove the navigation symbols from the bottom of all slides uncomment this line
      \AtBeginSection[]{
        \begin{frame}
        \vfill
        \centering
        \begin{beamercolorbox}[sep=8pt,center,shadow=true,rounded=true]{title}
          \usebeamerfont{title}\insertsectionhead\par%
        \end{beamercolorbox}
        \vfill
        \end{frame}

\begin{frame}<beamer>
\frametitle{Topic}
\tableofcontents[currentsection]
\end{frame}
      }
      \beamertemplateballitem

      \setbeameroption{show notes}
      \usepackage{listings}
      \lstset{numbers=none,language=[ISO]C++,tabsize=4,
  frame=single,
  basicstyle=\small,
  showspaces=false,showstringspaces=false,
  showtabs=false,
  keywordstyle=\color{blue}\bfseries,
  commentstyle=\color{red},
  }

      \institute[]{TU Delft}

       \subject{Quantum Computing}

\usepackage[utf8]{inputenc}
\usepackage[T1]{fontenc}
\usepackage{graphicx}
\usepackage{grffile}
\usepackage{longtable}
\usepackage{wrapfig}
\usepackage{rotating}
\usepackage[normalem]{ulem}
\usepackage{amsmath}
\usepackage{textcomp}
\usepackage{amssymb}
\usepackage{capt-of}
\usepackage{hyperref}
\usepackage{color}
\usepackage{minted}
\usepackage{color}
\usepackage{minted}
\usepackage{parskip}
\usetheme{default}
\author{Daniel Moreno Manzano}
\date{\today}
\title{}
\hypersetup{
 pdfauthor={Daniel Moreno Manzano},
 pdftitle={},
 pdfkeywords={},
 pdfsubject={},
 pdfcreator={Emacs 25.1.1 (Org mode 9.0.5)}, 
 pdflang={English}}
\begin{document}

\begin{frame}{Outline}
\tableofcontents
\end{frame}

\begin{frame}[fragile,label={sec:org639c711}]{qbench}
 \begin{block}{Algorithms analysis}
\alert{Note}: in the OPENQASM source files there are six algorithms using rotations along the z axis in any degree (QFT, Ising and Ground State Estimation).
Those are not translated for now.


\begin{table}[htbp]
\caption{\label{tab:org124c51e}
Algorithms analysis}
\centering
\tiny
\begin{tabular}{|p{3cm}|p{1.5cm}|p{1cm}|p{1cm}|p{7.5cm}|}
\hline
Algorithm & No. qubits & No. gates & Source & Expected Behaviour\\
\hline
 &  &  &  & \\
\texttt{0410184\_169} & 14 & 211 & RevLib & A form of integer adder\\
\texttt{3\_17\_13} & 3 & 36 & RevLib & worst case scenario for the approach presented in \cite{Miller_2003}\\
\texttt{4\_49\_16} & 5 & 217 & RevLib & worst case scenario for the approach presented in \cite{Miller_2003}\\
\texttt{4gt10-v1\_81} & 5 & 148 & RevLib & Checks, if input is greater than 10. (Note: Embedding with constant 1)\\
\texttt{4gt11\_82} & 5 & 27 & RevLib & Checks, if input is greater than 11. (Note: Embedding with constant 1)\\
\texttt{4gt11\_83} & 5 & 23 & RevLib & Checks, if input is greater than 11. (Note: Embedding with constant 1)\\
\texttt{4gt11\_84} & 5 & 18 & RevLib & Checks, if input is greater than 11. (Note: Embedding with constant 1)\\
\texttt{4gt12-v0\_86} & 6 & 251 & RevLib & Checks, if input is greater than 12. (Note: Embedding with constant 1)\\
\texttt{4gt12-v0\_87} & 6 & 247 & RevLib & Checks, if input is greater than 12. (Note: Embedding with constant 1)\\
\texttt{4gt12-v0\_88} & 6 & 194 & RevLib & Checks, if input is greater than 12. (Note: Embedding with constant 1)\\
\texttt{4gt12-v1\_89} & 6 & 228 & RevLib & Checks, if input is greater than 12. (Note: Embedding with constant 1)\\
\texttt{4gt13-v1\_93} & 5 & 68 & RevLib & Checks, if input is greater than 13. (Note: Embedding with constant 1)\\
\texttt{4gt13\_90} & 5 & 107 & RevLib & Checks, if input is greater than 13. (Note: Embedding with constant 1)\\
\texttt{4gt13\_91} & 5 & 103 & RevLib & Checks, if input is greater than 13. (Note: Embedding with constant 1)\\
\texttt{4gt13\_92} & 5 & 66 & RevLib & Checks, if input is greater than 13. (Note: Embedding with constant 1)\\
\texttt{4gt4-v0\_72} & 6 & 258 & RevLib & Checks, if input is greater than 4. (Note: Embedding with constant 1)\\
\texttt{4gt4-v0\_73} & 6 & 395 & RevLib & Checks, if input is greater than 4. (Note: Embedding with constant 1)\\
\texttt{4gt4-v0\_78} & 6 & 235 & RevLib & Checks, if input is greater than 4. (Note: Embedding with constant 1)\\
\texttt{4gt4-v0\_79} & 6 & 231 & RevLib & Checks, if input is greater than 4. (Note: Embedding with constant 1)\\
\texttt{4gt4-v0\_80} & 6 & 179 & RevLib & Checks, if input is greater than 4. (Note: Embedding with constant 1)\\
\texttt{4gt4-v1\_74} & 6 & 273 & RevLib & Checks, if input is greater than 4. (Note: Embedding with constant 1)\\
\texttt{4gt5\_75} & 5 & 83 & RevLib & Checks, if input is greater than 5. (Note: Embedding with constant 1)\\
\texttt{4gt5\_76} & 5 & 91 & RevLib & Checks, if input is greater than 5. (Note: Embedding with constant 1)\\
\texttt{4gt5\_77} & 5 & 131 & RevLib & Checks, if input is greater than 5. (Note: Embedding with constant 1)\\
\texttt{4mod5-bdd\_287} & 7 & 70 & RevLib & Performs a modulo operation of the binary encoding of the input and the constant 5\\
\texttt{4mod5-v0\_18} & 5 & 69 & RevLib & Performs a modulo operation of the binary encoding of the input and the constant 5\\
\texttt{4mod5-v0\_19} & 5 & 35 & RevLib & Performs a modulo operation of the binary encoding of the input and the constant 5\\
\texttt{4mod5-v0\_20} & 5 & 20 & RevLib & Performs a modulo operation of the binary encoding of the input and the constant 5\\
\texttt{4mod5-v1\_22} & 5 & 21 & RevLib & Performs a modulo operation of the binary encoding of the input and the constant 5\\
\texttt{4mod5-v1\_23} & 5 & 69 & RevLib & Performs a modulo operation of the binary encoding of the input and the constant 5\\
\texttt{4mod5-v1\_24} & 5 & 36 & RevLib & Performs a modulo operation of the binary encoding of the input and the constant 5\\
\texttt{4mod7-v0\_94} & 5 & 162 & RevLib & Performs a modulo operation of the binary encoding of the input and the constant 7\\
\texttt{4mod7-v1\_96} & 5 & 164 & RevLib & Performs a modulo operation of the binary encoding of the input and the constant 7\\
\texttt{9symml\_195} & 11 & 34881 & RevLib & \\
\texttt{C17\_204} & 7 & 467 & RevLib & \\
\texttt{adr4\_197} & 13 & 3439 & RevLib & \\
\texttt{aj-e11\_165} & 5 & 151 & RevLib & \\
\texttt{alu-bdd\_288} & 7 & 84 & RevLib & BDD-based synthesis of reversible logic for large functions based on \cite{Wille_2009}\\
\texttt{alu-v0\_26} & 5 & 84 & RevLib & Arithmetic Logic Unit (ALU)\\
\texttt{alu-v0\_27} & 5 & 36 & RevLib & Arithmetic Logic Unit (ALU)\\
\texttt{alu-v1\_28} & 5 & 37 & RevLib & Arithmetic Logic Unit (ALU)\\
\texttt{alu-v1\_29} & 5 & 37 & RevLib & Arithmetic Logic Unit (ALU)\\
\texttt{alu-v2\_30} & 6 & 504 & RevLib & Arithmetic Logic Unit (ALU)\\
\texttt{alu-v2\_31} & 5 & 451 & RevLib & Arithmetic Logic Unit (ALU)\\
\texttt{alu-v2\_32} & 5 & 163 & RevLib & Arithmetic Logic Unit (ALU)\\
\texttt{alu-v2\_33} & 5 & 37 & RevLib & Arithmetic Logic Unit (ALU)\\
\texttt{alu-v3\_34} & 5 & 52 & RevLib & Arithmetic Logic Unit (ALU)\\
\texttt{alu-v3\_35} & 5 & 37 & RevLib & Arithmetic Logic Unit (ALU)\\
\texttt{alu-v4\_36} & 5 & 115 & RevLib & Arithmetic Logic Unit (ALU)\\
\texttt{alu-v4\_37} & 5 & 37 & RevLib & Arithmetic Logic Unit (ALU)\\
\texttt{clip\_206} & 14 & 33827 & RevLib & \\
\texttt{cm152a\_212} & 12 & 1221 & RevLib & \\
\texttt{cm42a\_207} & 14 & 1776 & RevLib & \\
\texttt{cm82a\_208} & 8 & 650 & RevLib & \\
\texttt{cm85a\_209} & 14 & 11414 & RevLib & \\
\texttt{cnt3-5\_179} & 16 & 175 & RevLib & A reversible ternary counter with bit-width 5\\
\texttt{cnt3-5\_180} & 16 & 485 & RevLib & A reversible ternary counter with bit-width 5\\
\texttt{co14\_215} & 15 & 17936 & RevLib & \\
\texttt{con1\_216} & 9 & 954 & RevLib & \\
\texttt{cycle10\_2\_110} & 12 & 6050 & RevLib & \\
\texttt{dc1\_220} & 11 & 1914 & RevLib & \\
\texttt{dc2\_222} & 15 & 9462 & RevLib & \\
\texttt{decod24-bdd\_294} & 6 & 73 & RevLib & BDD-based synthesis of reversible logic for large functions based on \cite{Wille_2009}\\
\texttt{decod24-enable\_126} & 6 & 338 & RevLib & 2 to 4 binary decoder with enable\\
\texttt{decod24-v0\_38} & 4 & 51 & RevLib & 2 to 4 binary decoder\\
\texttt{decod24-v1\_41} & 5 & 85 & RevLib & 2 to 4 binary decoder\\
\texttt{decod24-v2\_43} & 4 & 52 & RevLib & 2 to 4 binary decoder\\
\texttt{decod24-v3\_45} & 5 & 150 & RevLib & 2 to 4 binary decoder\\
\texttt{dist\_223} & 13 & 38046 & RevLib & \\
\texttt{ex-1\_166} & 3 & 19 & RevLib & \\
\texttt{ex1\_226} & 6 & 7 & RevLib & \\
\texttt{ex2\_227} & 7 & 631 & RevLib & \\
\texttt{ex3\_229} & 6 & 403 & RevLib & \\
\texttt{f2\_232} & 8 & 1206 & RevLib & \\
\texttt{graycode6\_47} & 6 & 5 & RevLib & The Graycode function transforms the number \(x\) (with \(0 \le x<26-1\)) into the graycode.\\
\texttt{ham15\_107} & 15 & 8763 & RevLib & This function realizes the hamming code of a 15 variables input.\\
\texttt{ham3\_102} & 3 & 20 & RevLib & This function realizes the hamming code of a 3 variables input.\\
\texttt{ham7\_104} & 7 & 320 & RevLib & This function realizes the hamming code of a 7 variables input.\\
\texttt{hwb4\_49} & 5 & 233 & RevLib & This function describes the hidden weighted bit function (HWB) over 4 variables. HWB seems to be the simplest function with exponential OBDD size \cite{Bollig_1999}\\
\texttt{hwb5\_53} & 6 & 1336 & RevLib & This function describes the hidden weighted bit function (HWB) over 5 variables. HWB seems to be the simplest function with exponential OBDD size \cite{Bollig_1999}\\
\texttt{hwb6\_56} & 7 & 6723 & RevLib & This function describes the hidden weighted bit function (HWB) over 6 variables. HWB seems to be the simplest function with exponential OBDD size \cite{Bollig_1999}\\
\texttt{hwb7\_59} & 8 & 24379 & RevLib & This function describes the hidden weighted bit function (HWB) over 7 variables. HWB seems to be the simplest function with exponential OBDD size \cite{Bollig_1999}\\
\texttt{hwb8\_113} & 9 & 69380 & RevLib & This function describes the hidden weighted bit function (HWB) over 8 variables. HWB seems to be the simplest function with exponential OBDD size \cite{Bollig_1999}\\
\texttt{hwb9\_119} & 10 & 207775 & RevLib & This function describes the hidden weighted bit function (HWB) over 9 variables. HWB seems to be the simplest function with exponential OBDD size \cite{Bollig_1999}\\
\texttt{inc\_237} & 16 & 10619 & RevLib & \\
\texttt{life\_238} & 11 & 22445 & RevLib & Identical to "life\(_{\text{min}}\)" function\\
\texttt{majority\_239} & 7 & 612 & RevLib & \\
\texttt{max46\_240} & 10 & 27126 & RevLib & \\
\texttt{miller\_11} & 3 & 50 & RevLib & This function describes the Toffoli gate (Miller gate)\\
\texttt{mini-alu\_167} & 5 & 288 & RevLib & Simple ALU\\
\texttt{mini\_alu\_305} & 10 & 173 & RevLib & \\
\texttt{mini\_alu\_305} & 10 & 173 & RevLib & Simple ALU\\
\texttt{misex1\_241} & 15 & 4813 & RevLib & \\
\texttt{mlp4\_245} & 16 & 18852 & RevLib & A single digit BCD modulo-10 counter\\
\texttt{mod10\_171} & 5 & 244 & RevLib & A single digit BCD modulo-10 counter\\
\texttt{mod10\_176} & 5 & 178 & RevLib & This function realizes addition modulo 5\\
\texttt{mod5adder\_127} & 6 & 555 & RevLib & \\
\texttt{mod5d1\_63} & 5 & 22 & RevLib & \\
\texttt{mod5d2\_64} & 5 & 53 & RevLib & \\
\texttt{mod5mils\_65} & 5 & 35 & RevLib & \\
\texttt{mod8-10\_177} & 6 & 440 & RevLib & A single digit BCD counter that counts modulo 8 if e=0 and modulo 10 if e=1\\
\texttt{mod8-10\_178} & 6 & 342 & RevLib & A single digit BCD counter that counts modulo 8 if e=0 and modulo 10 if e=1\\
\texttt{one-two-three-v0\_97} & 5 & 290 & RevLib & This function has 3 inputs and three outputs. The first output is assigned to 1 if in total 1 input is assigned to 1. The second output is assigned to 1 if in total 2 inputs are assigned to 1. The third output is assigned to 1 if in total all 3 inputs are assigned to 1.\\
\texttt{one-two-three-v0\_98} & 5 & 146 & RevLib & This function has 3 inputs and three outputs. The first output is assigned to 1 if in total 1 input is assigned to 1. The second output is assigned to 1 if in total 2 inputs are assigned to 1. The third output is assigned to 1 if in total all 3 inputs are assigned to 1.\\
\texttt{one-two-three-v1\_99} & 5 & 132 & RevLib & This function has 3 inputs and three outputs. The first output is assigned to 1 if in total 1 input is assigned to 1. The second output is assigned to 1 if in total 2 inputs are assigned to 1. The third output is assigned to 1 if in total all 3 inputs are assigned to 1.\\
\texttt{one-two-three-v2\_100} & 5 & 69 & RevLib & This function has 3 inputs and three outputs. The first output is assigned to 1 if in total 1 input is assigned to 1. The second output is assigned to 1 if in total 2 inputs are assigned to 1. The third output is assigned to 1 if in total all 3 inputs are assigned to 1.\\
\texttt{one-two-three-v3\_101} & 5 & 70 & RevLib & This function has 3 inputs and three outputs. The first output is assigned to 1 if in total 1 input is assigned to 1. The second output is assigned to 1 if in total 2 inputs are assigned to 1. The third output is assigned to 1 if in total all 3 inputs are assigned to 1.\\
\texttt{plus63mod4096\_163} & 13 & 128744 & RevLib & Add 63 to the input modulo 4096\\
\texttt{plus63mod8192\_164} & 14 & 187112 & RevLib & Add 63 to the input modulo 8192\\
\texttt{pm1\_249} & 14 & 1776 & RevLib & \\
\texttt{radd\_250} & 13 & 3213 & RevLib & \\
\texttt{rd32-v0\_66} & 4 & 34 & RevLib & Counts the number of ones in the input.\\
\texttt{rd32-v1\_68} & 4 & 36 & RevLib & Counts the number of ones in the input.\\
\texttt{rd32\_270} & 5 & 84 & RevLib & Counts the number of ones in the input.\\
\texttt{rd53\_130} & 7 & 1043 & RevLib & Counts the number of ones in the input.\\
\texttt{rd53\_131} & 7 & 469 & RevLib & Counts the number of ones in the input.\\
\texttt{rd53\_133} & 7 & 580 & RevLib & Counts the number of ones in the input.\\
\texttt{rd53\_135} & 7 & 296 & RevLib & Counts the number of ones in the input.\\
\texttt{rd53\_138} & 8 & 132 & RevLib & Counts the number of ones in the input.\\
\texttt{rd53\_251} & 8 & 1291 & RevLib & Counts the number of ones in the input.\\
\texttt{rd53\_311} & 13 & 275 & RevLib & Counts the number of ones in the input.\\
\texttt{rd73\_140} & 10 & 230 & RevLib & Counts the number of ones in the input.\\
\texttt{rd73\_252} & 10 & 5321 & RevLib & Counts the number of ones in the input.\\
\texttt{rd84\_142} & 15 & 343 & RevLib & Counts the number of ones in the input.\\
\texttt{rd84\_253} & 12 & 13658 & RevLib & Counts the number of ones in the input.\\
\texttt{root\_255} & 13 & 17159 & RevLib & \\
\texttt{sao2\_257} & 14 & 38577 & RevLib & \\
\texttt{sf\_274} & 6 & 781 & RevLib & Sample function generated using ESOPSolver v.0 to demonstrate the use of ordering product terms and modifying variable polarity indices.\\
\texttt{sf\_276} & 6 & 778 & RevLib & Sample function generated using ESOPSolver v.0 to demonstrate the use of ordering product terms and modifying variable polarity indices.\\
\texttt{sqn\_258} & 10 & 10223 & RevLib & \\
\texttt{sqrt8\_260} & 12 & 3009 & RevLib & \\
\texttt{squar5\_261} & 13 & 1993 & RevLib & \\
\texttt{square\_root\_7} & 15 & 7630 &  & \\
\texttt{sym10\_262} & 12 & 64283 & RevLib & Symmetric function. This is a 10 inputs and 1 output function\\
\texttt{sym6\_145} & 7 & 3888 & RevLib & Symmetric function. This is a 6 inputs and 1 output function. The output is assigned to one iff the number of ones in the input is 2, 3, or 4\\
\texttt{sym6\_316} & 14 & 270 & RevLib & Symmetric function. This is a 6 inputs and 1 output function. The output is assigned to one iff the number of ones in the input is 2, 3, or 4\\
\texttt{sym9\_146} & 12 & 328 & RevLib & Symmetric function. This is a 9 inputs and 1 output function. The output is assigned to one iff the number of ones in the input is 3, 4, 5 or 6\\
\texttt{sym9\_148} & 10 & 21504 & RevLib & Symmetric function. This is a 9 inputs and 1 output function. The output is assigned to one iff the number of ones in the input is 3, 4, 5 or 6\\
\texttt{sym9\_193} & 11 & 34881 & RevLib & Symmetric function. This is a 9 inputs and 1 output function. The output is assigned to one iff the number of ones in the input is 3, 4, 5 or 6\\
\texttt{sys6-v0\_111} & 10 & 215 & RevLib & \texttt{sym6}\\
\texttt{urf1\_149} & 9 & 184864 & RevLib & Unstructured Reversible Function 1\\
\texttt{urf1\_278} & 9 & 54766 & RevLib & Unstructured Reversible Function 1\\
\texttt{urf2\_152} & 8 & 80480 & RevLib & Unstructured Reversible Function 2\\
\texttt{urf2\_277} & 8 & 20112 & RevLib & Unstructured Reversible Function 2\\
\texttt{urf3\_155} & 10 & 423488 & RevLib & Unstructured Reversible Function 3\\
\texttt{urf3\_279} & 10 & 125362 & RevLib & Unstructured Reversible Function 3\\
\texttt{urf4\_187} & 11 & 512064 & RevLib & Unstructured Reversible Function 4\\
\texttt{urf5\_158} & 9 & 164416 & RevLib & Unstructured Reversible Function 5\\
\texttt{urf5\_280} & 9 & 49829 & RevLib & Unstructured Reversible Function 5\\
\texttt{urf6\_160} & 15 & 171840 & RevLib & Unstructured Reversible Function 6\\
\texttt{wim\_266} & 11 & 986 & RevLib & \\
\texttt{xor5\_254} & 6 & 7 & RevLib & \\
\texttt{z4\_268} & 11 & 3073 & RevLib & \\
\hline
\end{tabular}
\end{table}


\begin{block}{Classification}
Based on the RevLib algorithm classification

\begin{center}    
\begin{tikzpicture}[sibling distance=3pt]
  \tikzset{grow'=right,level distance=130pt}
  \tikzset{execute at begin node=\strut}
  \tikzset{every tree node/.style={align=center,anchor=base west}}
  %% \tikzset{edge from parent/.style={draw,
  %%     edge from parent path={(\tikzparentnode.east)
  %%       -- +(0,-8pt)
  %%       |- (\tikzchildnode)}}}
  \tikzset{level 2/.style={level distance=120pt}}
  %% \tikzset{level 3/.style={level distance=120pt}}
  %% \tikzset{level 4/.style={level distance=100pt}}
  %% \tikzset{frontier/.style={distance from root=500pt}}
  \Tree [.{Benchmarks Classes}
    {Quantum Gates}
    {Encoding Functions}
    {Arithmetic Functions}
    {Miscellaneous}
    ]
    \end{tikzpicture}
\end{center}

\begin{block}{Quantum gates}
\begin{itemize}
\item Miller Gate
\end{itemize}
\end{block}

\begin{block}{Encoding Functions}
\begin{itemize}
\item Decod24
\item Decod24 with enable
\item Graycode
\item Hamming Code
\end{itemize}
\end{block}

\begin{block}{Arithmetic Functions}
\begin{itemize}
\item ALUs
\item 0410184
\item 1-bit Adder / rd32
\item 4 greater than 10
\item 4 greater than 11
\item 4 greater than 12
\item 4 greater than 13
\item 4 greater than 4
\item 4 greater than 5
\item 4 modulo 7
\item Check 4 modulo 5
\item dist
\item majority
\item max46
\item mlp4
\item mod5adder
\item mod5d1
\item mod5d2
\item mod5mils
\item Modulo 8/10 Counter
\item One-Two-Three Counter
\item plus127mod8192
\item plus63mod4096
\item plus63mod8192
\item radd
\item rd32
\item rd53
\item rd73
\item rd84
\item root
\item sqn
\item sqrt8
\item squar5
\item xor5
\item z4
\end{itemize}
\end{block}

\begin{block}{Miscellaneous}
\begin{itemize}
\item 9symml
\item adr4
\item aj-e11
\item C17
\item clip
\item cm152a
\item cm42a
\item cm82a
\item cm85a
\item co14
\item con1
\item cycle10\(_{\text{2}}\)
\item dc1
\item dc2
\item ex-1
\item ex1
\item ex2
\item ex3
\item f2
\item inc
\item life
\item misex1
\item pm1
\item sao2
\item Unstructured Reversible Functions. The urf benchmarks include six functions with various sizes that have no regular structure in their specifications. These functions are introduced to evaluate the behavior of a synthesis algorithm in synthesizing completely unstructured functions. The urf functions were originally proposed by M. Saeedi.
\item 3\(_{\text{17}}\)
\item 4\(_{\text{49}}\)
\item Hidden Weighted Bit
\item sym10
\item sym6
\item sym9
\end{itemize}
\end{block}
\end{block}


\begin{block}{Statistics}
\begin{itemize}
\item Highest amount of gates: \texttt{urf} has a maximum of 512064 gates followed by the \texttt{hwb9\_119} with 207775 gates
\item Number of different algorithms: 83
\item More details in \href{benchmarks\_profile/README.org}{Benchmarks Profile}
\end{itemize}
\end{block}
\end{block}

\begin{block}{OPENQASM to OpenQL translation}
TODO\ldots{}
\end{block}

\begin{block}{OpenQL compilation}
The configuration file used to compile the OpenQL algorithms is \url{hardware\_config\_cc\_light.json}
\end{block}
\end{frame}
\end{document}
